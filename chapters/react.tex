\chapter{React – A JavaScript library for building user interfaces}
\label{cha:react}

This chapter introduces the reader to the very popular and widespread front end library called ReactJS. It explains an essential part of React -- its rendering cycle -- which the reader needs to understand at least on a high level to be able to follow upcoming explanations of how the thesis project was implemented. Additionally, the paper elaborates the difference of how React uses declarative code to render data, whereas D3 uses an imperative API to render its data.

\section{Introduction to ReactJS}

The easiest way to find information about React is to visit its official website on \cite{React}. There is a statement up front that says: \begin{quote}\begin{english}React is a JavaScript library for building user interfaces.\end{english}\end{quote} which describes React very well. A Facebook engineer called Jordan Walke founded the library in 2011 as presented in \cite[05:30]{ReactFoundingVideo}. Walke wanted to create a library that would improve the code quality of their internal tool called ''Facebook ads''. Up until then, Facebook continued to develop and use React internally, but since the year 2013, the project is entirely open source. Since the initial open-source release up until now, technical engineers of Facebook but also the React open source community itself maintain the library. In late 2017 Facebook even changed React's BSD license to the MIT license, which is even better for the React community, as the MIT license has lesser restrictions than the BSD license.

According to \cite{React} React sees itself as a declarative and component-based library. However, a question might come to mind: ''What exactly does it mean for a library to be declarative and component-based?''. The answer to this question might be more straightforward than initially anticipated. In \cite{lloyd1994practical} declarative programming is described as a programming pattern, that expresses the logic of a program without describing its control flow. This means that the actual code describes what has to be computed not exactly how it should be done exactly. Declarative programming can be understood as a layer of abstraction, that makes code easier to understand for readers of the code. Declarative programming is therefore very different from the imperative programming pattern described in chapter \ref{cha:d3js}. React is a declarative library since its API lets developers describe how the application has to look like at any given data variation. Further information about React's API can be found in section \ref{sec:reactApi} though.

Enabling developers to create a highly component oriented architecture in their software is a fundamental aspect of React as well. Using a component-based library can increase productivity a tremendous amount. However, what does it mean for a library to favor component based architecture? After the initial setup of some boilerplate code, React makes it exceptionally easy to reuse existing components in the codebase to allow even faster development cycles. Once standard input components like buttons or text fields and layout components like page or header components are implemented, they can be reused throughout the whole app; thus significant progress can be achieved in a very short amount of time. Components can be manipulated by passing different properties, which might result in different presentation results of the components. More in-depth information about how React handles components and its props can be found in section \ref{sec:reactApi}.
    
The documentation in \cite{React} claims, that the library makes use of a so called ''virtual DOM''. This means, that React will keep track of its state data to prevent unnecessary writes to the actual DOM object. JavaScript performs exceptionally well when handling pure JavaScript objects in memory. Keeping the DOM tree of the application in the JavaScript engine's heap as a representation of objects enables React to apply data updates this so-called virtual DOM, then diff the newly applied data with the old tree to then being able to decide what DOM nodes need to be rewritten. Writing or committing the real DOM is the most valuable work in the browser, so React tries to keep those actions to a minimum. The diffing algorithm is called the ''reconciliation algorithm''. It would go out of the scope of this paper to go more in depth of the algorithm, so it is recommended to read about React's reconciliation algorithm in its documentation \cite{React}.

\section{Explaining the React API}
\label{sec:reactApi}

\section{The difference between imperative and declarative APIs}


