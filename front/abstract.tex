\chapter{Abstract}

\emph{D3} is one of the most powerful data visualization libraries in the \emph{JavaScript} environment. Because the development of the library started in 2011 and although the library went through several re-writes, the API is still relatively outdated. As a consequence, projects that use \emph{D3} become unmaintainable relatively quickly as \emph{D3's} API was not designed to realize extensive enterprise projects.

To avoid hardly maintainable code bases, the idea to combine \emph{D3} with a well-known and widespread \emph{JavaScript} library called \emph{React} came to mind. The claim that developer experience improves by writing \emph{React} code to use \emph{D3} features might be subjective. Thus the research question of this thesis is if the combination of \emph{React} and \emph{D3} can be achieved without sacrificing render performance.

Hence the master's thesis introduces the reader to three combination prototypes which were developed as a part of the thesis project. The thesis not only explains implementation details but also compares the advantages and disadvantages of the prototypes. However, the central research aspect of this thesis focuses on the performance comparison of the prototypes and shows if a combination of \emph{React} and \emph{D3} with performance results comparable to a native \emph{D3} implementation can be achieved.

%This fact is quite noticeable when multiple developers have to maintain a big codebase of D3 production code. 