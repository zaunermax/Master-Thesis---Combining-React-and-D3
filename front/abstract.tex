\chapter{Abstract}

With D3 being one of the most powerful data visualization libraries in the JavaScript environment, it is shocking, how bad developer experience can be when writing extensive amounts of D3 code. D3 is a reasonably old library as its development started in early 2011 when the scripting language JavaScript was in a completely different state than it is today. In that era, most developers would have never even thought about utilizing JavaScript as the primary technology to realize enterprise projects. Even tough D3 went through a few significant rewrites and got refactored multiple times; the base API is still quite similar to the original API. 

Since the API sometimes might be hard to use in production environments, the idea to combine the D3 library with a well-known and widespread JavaScript library called React came to mind. The claim that developer experience improves by using React code to use D3 features might be subjective, and as a consequence, the research question of this thesis is if the combination of React and D3 can be achieved without sacrificing render performance. 

Hence this paper introduces the reader to the combination of the prototypes which were developed as a part of the thesis project. The thesis not only explains implementation details but also compares the advantages and disadvantages of the prototypes. However, the central research aspect of this paper focuses on the performance comparison of the prototypes and shows if a combination with comparable performance as a native D3 implementation can be achieved.

%This fact is quite noticeable when multiple developers have to maintain a big codebase of D3 production code. 