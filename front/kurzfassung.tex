\chapter{Kurzfassung}

\begin{german}
Da D3 eine der leistungsfähigsten und umfangreichsten Datenvisualisierungsbibliotheken im \mbox{JavaScript}-Umfeld ist, ist es schockierend, wie schlecht die Entwicklererfahrung beim Schreiben großer Mengen an D3-Code sein kann. D3 ist eine ziemlich alte Bibliothek, da ihre Entwicklung bereits Anfang 2011 begann, als die Skriptsprache \mbox{JavaScript} in einem völlig anderen Zustand war als heute. In jener Zeit hätten Entwickler nie daran gedacht, \mbox{JavaScript} als die primäre Technologie zur Realisierung von Enterprise Projekten zu nutzen. Obwohl D3 ein paar wichtige Neuschreibungen durchlief und mehrfach überarbeitet wurde, ist die derzeitige API der ursprünglichen noch recht ähnlich. 

Da die API in Produktionsumgebungen manchmal schwer zu verwenden ist, kam die Idee in den Sinn, die D3-Bibliothek mit einer bekannten und weit verbreiteten JavaScript-Bibliothek namens React zu kombinieren. Die Behauptung, dass sich die Entwicklererfahrung durch die Verwendung von React-Code zur Verwendung von D3-Funktionen verbessert, könnte subjektiv sein. Als Folge davon ist die Forschungsfrage dieser Arbeit, ob die Kombination von React und D3 erreicht werden kann, ohne die Renderleistung zu beeinträchtigen. 

Daher erklärt die Arbeit den Leser in die Kombination der Prototypen ein, die im Rahmen des Promotionsprojekts entwickelt wurden. Es werden nicht nur die Details der Implementierung erklärt, sondern auch die Vor- und Nachteile der Prototypen werden verglichen. Der zentrale Forschungsaspekt dieses Papiers konzentriert sich jedoch auf den Leistungsvergleich der Prototypen und zeigt, ob eine Kombination mit vergleichbarer Leistung wie eine native D3-Implementierung erreicht werden kann.
\end{german}