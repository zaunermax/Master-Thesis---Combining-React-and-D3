\chapter{Kurzfassung}

\begin{german}
D3 ist eine der umfangreichsten und leistungsfähigsten Datenvisualisierungsbibliotheken im JavaScript-Umfeld. Da die Entwicklung der Bibliothek im Jahr 2011 begann und obwohl die Bibliothek mehrere Neuschreibungen durchlaufen hat, ist die API noch relativ veraltet. Infolgedessen werden Projekte, die D3 verwenden, relativ schnell nicht mehr wartbar, da die API von D3 nicht für die Realisierung umfangreicher Unternehmensprojekte konzipiert wurde.

Um kaum wartbaren Code zu vermeiden, kam die Idee in den Sinn, D3 mit einer bekannten und weit verbreiteten JavaScript-Bibliothek namens React zu kombinieren. Die Behauptung, dass sich die Erfahrung des Entwicklers verbessert, wenn man React-Code schreibt, um D3-Funktionen zu verwenden, könnte subjektiv sein. Daher ist die Forschungsfrage dieser Arbeit, ob die Kombination von React und D3 erreicht werden kann, ohne die Renderleistung zu beeinträchtigen.

Daher erklärt die Arbeit den Leser in die Kombination der Prototypen ein, die im Rahmen des Promotionsprojekts entwickelt wurden. Es werden nicht nur die Details der Implementierung erklärt, sondern auch die Vor- und Nachteile der Prototypen werden verglichen. Der zentrale Forschungsaspekt dieses Papiers konzentriert sich jedoch auf den Leistungsvergleich der Prototypen und zeigt, ob eine Kombination mit vergleichbarer Leistung wie eine native D3-Implementierung erreicht werden kann.
\end{german}