\chapter{Abstract}

D3 is one of the most powerful data visualization libraries in the JavaScript environment. Because the development of the library started in 2011 and although the library went through several re-writes, the API is relatively outdated still. As a consequence, projects that use D3 become unmaintainable relatively quickly as D3's API was not designed to realize extensive enterprise projects.

To avoid hardly maintainable code bases, the idea to combine D3 with a well-known and widespread JavaScript library called React came to mind. The claim that developer experience improves by writing React code to use D3 features might be subjective. Thus the research question of this thesis is if the combination of React and D3 can be achieved without sacrificing render performance.

Hence this paper introduces the reader to the combination of the prototypes which were developed as a part of the thesis project. The thesis not only explains implementation details but also compares the advantages and disadvantages of the prototypes. However, the central research aspect of this paper focuses on the performance comparison of the prototypes and shows if a combination with comparable performance as a native D3 implementation can be achieved.

%This fact is quite noticeable when multiple developers have to maintain a big codebase of D3 production code. 