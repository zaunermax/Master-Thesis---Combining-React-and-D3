\chapter{Conclusion}
\label{cha:conclusion}

This chapter is the conclusion of all the findings of this master's thesis. Not only the prototypes are recapitulated, but also the test results are summarized. The most apparent finding emerging from this thesis is the fact that there is a possible combination of \emph{React} and \emph{D3} which only sacrifices a small proportion of performance to enable \emph{React} developers to use \emph{D3} functionalities, but write \emph{React} code.

\section{Prototypes}

Three prototypes were developed to implement a combination of \emph{React} and \emph{D3}. Every prototype has its advantages and disadvantages. They all pursue the same goal though, which is to provide a \emph{React} component that completely encapsulates the complete \emph{D3} force simulation functionality. The force graph component has to be usable like any other \emph{React} component, which means that the component has to be a declarative representation of state. To get a force simulation component that is a function of state, all prototypes have to \emph{React} to state changes accordingly.

\subsection{Pure D3 Prototype}

The first prototype is the pure \emph{D3} variant as introduced in section \ref{sec:pureD3prototype}. The component is a \emph{React} wrapper that renders a base element which \emph{D3} then hooks into. \emph{D3} is completely in charge of the whole simulation. Not only does \emph{D3} append and remove the nodes from the DOM but also \emph{D3} controls all node positions of the simulation. The prototype relies on \emph{React's} reconciliation algorithm, which never detects any change in the staticly rendered base element and always keeps the reference to the real SVG element intact.

The implementation is very similar to a \emph{D3} only implementation, as \emph{React} is only used to render the static base SVG element. Much imperative \emph{D3} code is used to append and remove nodes. Furthermore, the code contains a lot of \emph{D3's} imperative selection functions, which help to keep track of entering and exiting nodes to apply transitions to them. The complete presentation layer is handled by \emph{D3}, which includes not only style attributes but also the positions of the nodes and links.

A significant advantage of the pure \emph{D3} prototype is that its performance is close to if not the same as a native \emph{D3} implementation. The most notable disadvantage is that almost the entire code of the component is written in pure \emph{D3} code, which could lead to a poor developer experience if the component advances in the future. Also, no custom render functions can be used, as a custom mechanism would have to be added to enable user customized \emph{D3} code in the component. 

\subsection{Pure React Prototype}

Section \ref{sec:pureReactPrototype} introduces the pure \emph{React} prototype. \emph{React} is completely in charge of the presentational layer. As a consequence, not only the nodes' and links' positions but also their styling and node types are completely handled by \emph{React}. When the component mounts, \emph{D3's} force simulation is initialized with the data. The prototype then pulls the data via the tick function and sets it in its internal state. Due to the nature of \emph{React}, the component goes through a new render cycle to render the newly fetched node and link position. The complete cycle happens multiple times a second.

As mentioned before, the component's constructor is used to initialize \emph{D3} even before \emph{React} renders anything. The tick handling function pulls all node and link positions from the simulation every tick and sets it via \emph{React's} set state function every fraction of a second as often as \emph{D3's} tick function is called. The component then completely rerenders with the newly set node and link positions. Because the component's state updates every fraction of a second, component prop updates have to be distinguished from state updates. The \texttt{shouldComponentUpdate()} function is used to update the \emph{D3} force simulation only if the props have changed.

One of the most significant advantages of the prototype is that maintainers of the component can use declarative \emph{React} code to handle the complete presentation layer of the simulation. In addition to the styling and node type, also the position is entirely handled by \emph{React}. The fact that \emph{React} is used to handle the whole simulation tree is also the most considerable disadvantage because it implies that \emph{React} has to go through a complete rendering cycle every time the simulation's tick handler is called. Another advantage is that props can customize the prototype's render functions. However, it is quite complicated to achieve the full \emph{D3} functionality like dragging and enter and exit animations, as they have to be implemented from scratch. 

\subsection{React and D3 Hybrid Prototype}

Last but not least, there is the \emph{React} and \emph{D3} hybrid prototype, which is presented in section \ref{sub:D3AndReactHybrid}. The hybrid component works by initially rendering all the simulation's nodes via \emph{React} and then letting \emph{D3} select the already existing nodes to handle the nodes' positioning. \emph{React} handles the styling and node shape and \emph{D3} handles the node and link positions. The hybrid component uses all the lifecycle methods in \emph{React's} commit phase to being able to read the DOM nodes that have already been committed to the DOM. 

The implementation of the hybrid component is very straight forward. The \emph{React} lifecycle method \texttt{componentDidMount()} is used to initialize the \emph{D3} simulation and the \texttt{componentDidUpdate()} lifecycle method is used to update the \emph{D3} force simulation if the simulation data props have changed. The render method takes care of the complete simulation presentation except for the positions, which are handled in the tick updater function. The tick handler function is one of the few instances in the hybrid component where \emph{D3} code is used.

Being able to make the render functions pluggable is also a significant advantage of the hybrid component. Also, \emph{D3} functionality like dragging nodes, zooming, and adding transitions is easy to implement with the hybrid prototype. The only disadvantage might be that maintainers of the prototype would have to completely understand how the rendering process works to be able to contribute to the component.

\section{Performance Results}

This thesis aims to find a well-performing combination of \emph{React} and \emph{D3}. The research question "Can \emph{React} be combined with \emph{D3} without losing performance in the browser" unfortunately has to be answered with "\emph{React} can be combined with \emph{D3}, but there are some slight render performance losses". The results of the performance benchmarks clearly show how the newly invented hybrid component is always ahead of Uber's take on combining the two libraries. 

However, when looking at the performance numbers, research shows that every prototype can be used on high-end devices without notable performance issues. The low-end devices suffer from low calculation speeds and low monitor refresh rates in general. Even though the hybrid prototype is ahead of the pure \emph{React} component when measuring the raw performance, it is clear though that the performance differences are hardly measurable without knowing the exact numbers.

\section{Open Source}

Because the hybrid prototype is easily extensible with not only \emph{D3} but also \emph{React} functionality, the decision was made to open-source the component. The \emph{React} community lives off of free to use components that can be integrated into any project free of charge. The hybrid component was developed out of necessity for a better solution of combining \emph{React} and \emph{D3}, and maybe other developers find it useful as well.

\section{Final Thoughts}

Even though the performance numbers could have been better, the project is considered a success nevertheless due to the fact that the hybrid prototype's performance is not only on par or even superior to Uber's pure \emph{React} variant. The performance is worse than a native \emph{D3} implementation, but that is to be expected when two big libraries are combined. The hybrid's component API also turned out to be very useful as the storybook examples demonstrate. Being able to not only use the power of \emph{D3} but also to write almost pure declarative \emph{React} code to implement a force simulation is definitely considered a success.


% all in all big success, as i found an implementation that combines d3 and react and is super performant. Also open sourcing the thing is really cool as other developers can benefit from using the force component by coding their nodes in react as well.

% Developer experience

% Performance on par or superior

% Ease of development

% Open source component

% Comprehensive component API
