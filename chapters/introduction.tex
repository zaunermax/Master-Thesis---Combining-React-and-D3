\chapter{Introduction}
\label{cha:Introduction}

This chapter introduces the reader to the general concept of the master project. Not only should the initial situation and motives be clear after reading this section of the thesis, but also what the thesis project is about and what it tries to achieve.

\section{Problem description and motivation}

With \emph{D3} being one of the most powerful data visualization libraries in the \emph{JavaScript} environment, it is shocking how bad developer experience can be when writing extensive amounts of \emph{D3} code. \emph{D3} is a reasonably old library as its development started in early 2011 when the scripting language \emph{JavaScript} was in a completely different state than it is now. In that era, most developers would have never even thought about utilizing \emph{JavaScript} as the primary technology to use to realize big enterprise web projects. 

Even though \emph{D3} went through a few significant rewrites and got refactored multiple times throughout some major versions; the base API is still quite similar to the original API. This fact is quite noticeable when writing large amounts of \emph{D3} production code that---while still performing well---must be kept maintainable for multiple developers.

Since the API might be hard to use in production environments, the idea to combine the \emph{D3} library with a widely used \emph{JavaScript} library called \emph{React} came to mind. \emph{React} is currently used in big projects like \emph{Facebook}, \emph{Airbnb}, \emph{Netflix}, and \emph{Spotify}. The claim that developer experience improves by using \emph{D3} through \emph{React} might be subjective, but even if only a few developers are willing to use \emph{React} combined with \emph{D3} over native \emph{D3} in their software the thesis project is a valuable addition to the \emph{React} developer community. The exciting research aspect would then be if the combination is possible without losing render performance but still providing the convenience of writing \emph{React} code.

\section{Goals of the project}

The central research aspect of the master's thesis focuses on the performance aspect of the combination of \emph{React} and \emph{D3}, not how the \emph{React} version might provide a better developer experience than native \emph{D3} version. Subjective opinions can hardly be measured scientifically as the number of probands to measure developer experience on different library versions would have to be quite high to be able to get accurate heuristic results. Performance numbers, on the other hand, can easily be measured. A combination of the two libraries which would allow programmers to use some of \emph{D3's} functionality by writing declarative \emph{React} code without introducing any performance penalties would be a valuable software addition to the \emph{React} community. Methods of how a combination of the technologies can be achieved are elaborated in this thesis. Also, some already existing work is explained and analyzed.

The thesis aims to introduce the reader to the general concept of the two libraries---React and D3---which are combined in the thesis project. The master's thesis provides not only a general overview but also an introduction to the two libraries for an even better understanding of the thesis project. The first chapters explain the required knowledge to understand the performance discussion. Later chapters compare different approaches of possible implementations regarding the combination of the libraries. It is then easy to follow the discussion by already having the necessary knowledge to understand all required aspects of the mentioned technologies.

Another significant part of the thesis is the description of the thesis project itself. The combination of the libraries \emph{React} and \emph{D3} is a software project that was created out of a requirement. The goal is to create a software that allows developers to write declarative \emph{React} code and to avoid imperative \emph{D3} code while still using \emph{D3's} data visualization capabilities. Another goal is to keep performance losses at a minimum but still use the full extent of \emph{React's} features of writing web components. Of course, there are many ways to achieve the same goal; that's also why the thesis project provides three discussable prototypes. The thesis project chapter explains the implementation and functionality of each prototype. Ultimately, all prototypes are compared to each other.

The performance comparison of the prototypes is one of the most important parts of the thesis. Each prototype is tested on different devices and in different browsers, which generates some performance numbers that are also compared and discussed. While discussing raw performance numbers can give an insight into how the prototypes perform, they can also be essential to measure user experience. A vital part of the thesis is the explanation of user perception of animated content in the browser. The focus primarily lies on researching the threshold on which users do not percieve an animation as smooth anymore.

Last but not least, there is the goal of introducing the reader to the open source concept that is planned for the thesis project. Initially, a specific use-case was the reason to create the library that connects \emph{D3} and \emph{React}, but there are most certainly other developers that can make use of the thesis project as well. The thesis provides a general overview of how the public API of the technology is designed and how the project is published on a widely available package registry to allow for including the library in any project.
