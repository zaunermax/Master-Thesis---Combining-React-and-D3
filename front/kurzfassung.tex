\chapter{Kurzfassung}

\begin{german}
\emph{D3} ist eine der umfangreichsten und leistungsfähigsten Datenvisualisierungsbibliotheken im \emph{JavaScript}-Umfeld. Da die Entwicklung der Bibliothek im Jahr 2011 begann und obwohl die Bibliothek mehrere Neuschreibungen durchlaufen hat, ist die API noch relativ veraltet. Infolgedessen werden Projekte, die \emph{D3} verwenden, relativ schnell nicht mehr wartbar, da die API von \emph{D3} nicht für die Realisierung umfangreicher Unternehmensprojekte konzipiert wurde.

Um kaum wartbaren Code zu vermeiden, kam die Idee in den Sinn, \emph{D3} mit einer bekannten und weit verbreiteten \emph{JavaScript}-Bibliothek namens \emph{React} zu kombinieren. Die Behauptung, dass sich die Erfahrung des Entwicklers verbessert, wenn man React-Code schreibt, um \emph{D3}-Funktionen zu verwenden, könnte jedoch subjektiv sein. Daher ist die Forschungsfrage dieser Arbeit, ob die Kombination von \emph{React} und \emph{D3} erreicht werden kann, ohne die Renderleistung zu beeinträchtigen.

Die Masterarbeit führt den Leser in drei Kombinationsprototypen ein, die im Rahmen des Masterarbeitsprojekts entwickelt wurden. Es werden nicht nur die Details der Implementierung erklärt, sondern auch die Vor- und Nachteile der Prototypen werden verglichen. Der zentrale Forschungsaspekt diese Arbeit konzentriert sich jedoch auf den Leistungsvergleich der Prototypen und zeigt, ob eine Kombination mit vergleichbarer Leistung wie eine native \emph{D3}-Implementierung erreicht werden kann.
\end{german}